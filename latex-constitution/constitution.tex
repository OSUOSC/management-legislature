\documentclass[12pt,letterpaper]{article}

% Packages
\usepackage[margin=1in]{geometry}
\usepackage{graphicx}
\usepackage{titlesec}
\usepackage{titling}
\usepackage{enumitem}
\usepackage{fancyhdr}
\usepackage{xcolor}
\usepackage{pifont}
\usepackage{hanging}
\usepackage{xfrac}
\usepackage{setspace}
\usepackage{hyperref}
\usepackage{bookmark}

% Color for headings and accents
\definecolor{crimsonred}{RGB}{117,0,0}

% Custom heading formats
\renewcommand{\thesection}{\Roman{section}}
\renewcommand{\thesubsection}{\Roman{subsection}}
\renewcommand{\thesubsubsection}{\Roman{subsubsection}}
\titleformat{\section}
  {\normalfont\Large\bfseries\color{crimsonred}}
  {ARTICLE \thesection.}{0.5em}{}
\titleformat{\subsection}
  {\normalfont\large\bfseries}
  {SECTION \thesubsection.}{0.5em}{}
\titleformat{\subsubsection}
  {\normalfont\normalsize\bfseries}
  {CHAPTER \thesubsubsection.}{0.5em}{}
\setlist{noitemsep}

% Header and footer
\pagestyle{fancy}
\fancyhf{}
\renewcommand{\headrulewidth}{0.4pt}
\renewcommand{\footrulewidth}{0.4pt}
\fancyhead[R]{\slshape Constitution}
\fancyfoot[C]{\thepage}

% Document content starts here
\begin{document}

% Change bullet style
\renewcommand{\labelitemi}{\textcolor{crimsonred}{\ding{68}}}
\renewcommand{\labelitemii}{\textcolor{crimsonred}{\ding{67}}}
\renewcommand{\labelitemiii}{\textcolor{crimsonred}{\ding{66}}}
% \renewcommand{\labelitemii}{\textcolor{crimsonred}{$\blacktriangleright$}}
% \renewcommand{\labelitemiii}{\textcolor{crimsonred}{$\bullet$}}

% Title page
\begin{titlepage}
\begin{center}
\vspace*{1cm}

% Logo
\includegraphics[width=0.4\textwidth]{logo.png}\\[2cm]

% Title
{\LARGE\bfseries\color{crimsonred} CONSTITUTION\\[0.5cm]}
{\huge\bfseries\color{crimsonred} \underline{Open Source Club}\\[1cm]}

% Subtitle
{\large Last Revised: [Month Day, Year]}\\[3cm]

% Bottom of title page
\vfill
{\large Ohio State University}\\
{\large Columbus, OH}\\[1cm]
\end{center}
\end{titlepage}

% Table of contents
\tableofcontents
\clearpage

% Document proper starts here
\section*{PREAMBLE}
\addcontentsline{toc}{section}{PREAMBLE}

\begin{hangparas}{0.5in}{1}
  The Open Source Club at the Ohio State University has had an extensive role in OSU's
  tech-related student life, from sprouting into other clubs and events, to encouraging
  students and alumni alike to become a part of the open source community.

  The current leadership of this organization would like to extend gratitude to former
  members and staff; open source has had a rich history, both in and out of OSU's campus,
  and it is with the support of those who have come before us that we are able to exercise
  this passion so freely.

  A detailed record of past leadership for the Open Source Club at the Ohio State University
  is included in the complete legislature for the organization, which can be found at
  \href{https://legislature.osuosc.org}{legislature.osuosc.org}.

  Additionally, the following are the available direct links to the different ways in which
  this constitution can be read:

  \begin{itemize}
    \item \href{https://legislature.osuosc.org/constitution/preamble.html}{mdBook} (through
      your browser)
    \item \href{https://activities.osu.edu/involvement/student_organizations/find_a_student_org?i=1042}{PDF} (in the link provided to the right of "\textbf{Constitution}")
  \end{itemize}

  \textbf{This constitution is licensed as \href{https://creativecommons.org/licenses/by-sa/4.0/deed.en}{CreativeCommmons Attribution ShareAlike}. You are free to use it to create your own
  constitutions, both in format and in file, so long as you provide credit to the Open Source
Club at the Ohio State University for creating the original template and content, and allow
other individuals to do the same with whatever derivative constitution you create.}
\end{hangparas}

\section{Definition}

Article I of the Constitution for the Open Source Club at the Ohio State University specifies
the definition of the club, or, how the club ought to be referred to as, and the nature of the
club and its establishment.

\subsection{Name}

The name of this student organization is to be ``The Open Source Club at the Ohio State
University.'' ("The Open Source Club", "Open Source  Club", "OSC", "th(e/is) organization",
"th(e/is) student organization").

The following are appropriate shorthands that may be used throughout organization legislature
or in general reference to the organization:

\begin{itemize}
  \item The Open Source Club at OSU
  \item Open Source Club at OSU
  \item OSC at OSU
  \item OSC@tOSU
\end{itemize}

If the name is being provided in documentation or communications strictly within the
Ohio State University, suffixed terms "OSU" and "the Ohio State University" may be omitted
for brevity.

\subsection{Purpose}

This student organization will dedicate itself to helping promote, propogate, and standardize
the open source paradigm of technology deevelopment and use, both in software, and in other
works to which the open source paradigm is applicable and societally beneficial.

The organization will strive to excel in these three vectors by:

\begin{enumerate}[label=\textcolor{crimsonred}{\alph*.}, leftmargin=2em]
  \item Introducing the open source model of development to OSU students and the larger
    OSU/Columbus community;
  \item Showcasing projects developed under an open source development model (programs,
    games, websites, arts, etc.);
  \item Fostering interest in the open source community, and providing ways for members
    to involve themselves in it (meetings, talks, workshops, etc.);
  \item Creating a space in which members can engage in constructive discourse on the
    topics surrounding open source;
  \item Raising awareness of the value the open source paradigm can provide to students,
    members of our communities, and society at large;
  \item Inspire members to think critically about the technology they use, the way that
    technology interfaces and affects their lives, and the ways in which we can reclaim
    our agency and foster solidarity in our communities through a more sustainable
    relationship with technology and a more democratic paradigm of technology in our societies.
\end{enumerate}

\subsection{Non-Descrimination Policy}

This organization does not discriminate on the basis of age, ancestry, color, disability,
gender identity or expression, genetic information, HIV/AIDS status, military status,
national origin, race, religion, sex, sexual orientation, protected veteran status, or
any other bases under the law, in its activities, programs, admission, and employment.

In addition, this organization does not discriminate on the basis of academic standing,
program affiliation, nor personal political affiliation.

Finally, it is this organization's policy that these statements and policies extend to
- and will be required of - all individuals who choose to associate with this
organization; student, faculty, or otherwise.

\section{Membership}

In the pursuit of our stated purpose, membership in this student organization is extended
to all members of the OSU and Columbus community who wish to involve themselves - and
learn how to benefit from - open source. No fees or dues will ever be collected from any
member of the community wishing to involve themselves in the Open Source Club.

Recognizing the breadth of our potential member pool, we define the two types of
members as \textbf{active} and \textbf{passive}, and elaborate on the priviledges afforded
to each.

\subsection{Active Membership}

Active membership is awarded to any and all currently-enrolled students of the Ohio State
University, part- or full-time, who are in good standing with the organization, and who
display activity within the organization to a recognizable degree. This includes, but is
not limited to:

\begin{itemize}
  \item Those who attend two or more in-person meetings in a semester
  \item Those who are active in existing social groups (Discord, Revolt, Matrix, etc.)
    \begin{itemize}
      \item "Active" is loosely defined as contributing enough to online discussion so as
        to easily be precisely identified and recalled by members during in-person
        meetings, be it by username or reference to discussions of which the member in
        question was a part. A more precise definition is left to current officership
        discretion.
    \end{itemize}
  \item Other examples left to the officers' discretion.
\end{itemize}

Active membership affords members the privelege of continuing to participate in digital
social groups; being allowed to attend meetings and, to the discretion of associated
organizations, meeting collaborations.

Active membership additionally affords members the privilege of being able to run for,
and hold, executive officer positions; being able to vote for candidates for executive
roles; being able to participate in votes for the removal of an executive officer.

\subsection{Passive Membership}

Passive membership is awarded to any and all individuals who are in good standing
with the organization, and who display interest in the organization to a recognizable
degree. This includes, but is not limited to:

\begin{itemize}
  \item Those who subscribe to any existing mailing lists.
  \item Those who interact with any existing social media accounts.
  \item Those who are members of any existing social groups (Discord, Revolt, Matrix, etc.)
  \item Those who have expressed explicit interest in the club, but are unable to
    consistently interface with it due to personal constraints.
  \item Other examples left to the officers' discretion.
\end{itemize}

Passive membership affords members the privelege of continuing to participate in digital
social groups; being allowed to attend meetings and, to the discretion of associated
organizations, meeting collaborations.

\clearpage

\section{Procedures for Removing Members and Executive Officers}

It may be the case that a member of the organization, be they passive, active, or an
executive officer, has their behavior deemed incongruent with the stated goals of the
organization, in violation of the Ohio State University Standards, the Student Code of
Conduct, and/or the Non-Discrimination Policy as expressed in Article I, Section III, or
otherwise deemed unacceptable by the expectations of this organizations and its
membership. For these regrettable cases, and in the interest of fostering a productive,
accessible, and equitable community for the OSU community at large, a procedure for
removing the offending organization member is defined.

In calculating a non-unanimities (i.e. majorities or pluralities), abstentions from
active members are ignored. In calculating a unanimity, abstentions are equivalent to
nays.

Members removed from the organization through the procedures forthcoming lose their
status of good standing for two academic years.

\subsection{The Removal of a Non-Officer}

Non-officer members being considered for removal will be notified of this consideration,
whereafter a one-week voting period open to active members, passive members, and officers
will commence. The removal will be successful if:

\begin{itemize}
  \item A unanimous decision to remove is made among the officers.
  \item A simple majority decision to remove is made among passive and active members, separately.
\end{itemize}

\subsection{The Removal of an Officer}

Officer members being considered for removal will be notified of this consideration,
whereafter a two-week voting period open to active members will commence. The removal will
be successful if:

\begin{itemize}
  \item A \sfrac{2}{3} majority decision to remove is made among active members.
\end{itemize}

\section{Executive Officership}

In the pursuit of effective organization management, a structure of executive officership
is designed for active members of the organization to play a role in maintaining the
community and longevity thereof.

\clearpage

\subsection{Titles and Responsibilities}

Any role not marked with an asterisk (*) is considered supplementary, and its being
unfulfilled during an academic term is acceptable. In such an event, the available
executive officership will take on the stated responsibilities at their discretion. As a
default, any role not fulfilled will have their associated responsibilities taken on by
the President and Vice President as appropriate.

All Executive roles will have alternative titles that are to be recognized by past
leadership and advisors as valid. They exist so as to allow flexibility in how members of
the Executive Board of OSC represent their Officership to others, be that in professional
documents, in communicating with non-OSC parties, or otherwise.

\begin{itemize}
  \item \textbf{*President:} The President is the primary leader of the executive board,
    and oversees the large-scale success of the organization.
    \begin{itemize}
      \item The following is a non-exhaustive list of the expected responsibilities of
        the President of Open Source Club:
        \begin{itemize}
          \item Deciding meeting times and topic;
          \item Maintaining appropriate and welcoming behavior in OSC spaces (in-person,
            digital, and otherwise);
          \item Establishing and/or disestablishing social media accounts and virtual
            social groups;
          \item Representing OSC in matters with non-OSC parties;
          \item Ensuring detailed records and recommendations of OSC procedure are
            maintained;
          \item Ensuring all training and documents are completed and filled by all
            members of the Executive Board by time frames places by the Office of Student
            Life of the Ohio State University;
          \item Providing a model for what an effective, inclusive, and mindful member of
            OSC ought to behave like;
          \item Being mindful of the assets and events presently and in the future under
            OSC's purview.
        \end{itemize}
    \end{itemize}
\end{itemize}

\subsection{Voting and Tenure}

\subsubsection{Time Allotted to Self-Nomination and Voting}
\subsubsection{Precise Voting and Vote-Counting Procedure}
\subsubsection{Tenure}

% Signature section
\clearpage
\section*{RATIFICATION}
\addcontentsline{toc}{section}{RATIFICATION}

We, the undersigned, hereby ratify this Constitution on behalf of the Open Source Club at
the Ohio State University.

\vspace{1cm}
\begin{tabular}{p{3in}p{3in}}
\rule{3in}{0.5pt} & \rule{3in}{0.5pt} \\
President & Faculty Advisor \\
& \\
\rule{3in}{0.5pt} & \rule{3in}{0.5pt} \\
Vice President & Date of Ratification \\
\end{tabular}

\end{document}
