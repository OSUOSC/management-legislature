\documentclass[12pt,letterpaper]{article}

% Packages
\usepackage[margin=1in]{geometry}
\usepackage{graphicx}
\usepackage{transparent}
\usepackage{titlesec}
\usepackage{titling}
\usepackage{enumitem}
\usepackage{fancyhdr}
\usepackage{xcolor}
\usepackage{pifont}
\usepackage{hanging}
\usepackage{xfrac}
\usepackage{setspace}
\usepackage[hidelinks]{hyperref}
\usepackage{tocloft}
\usepackage{bookmark}

% Color for headings and accents
\definecolor{accent}{RGB}{117,0,0}

% Custom heading formats
\renewcommand{\thesection}{\Roman{section}}
\renewcommand{\thesubsection}{\Roman{subsection}}
\renewcommand{\thesubsubsection}{\Roman{subsubsection}}
\titleformat{\section}
  {\normalfont\Large\bfseries\color{accent}}
  {ARTICLE \thesection.}{0.5em}{}
\titleformat{\subsection}
  {\normalfont\large\bfseries}
  {SECTION \thesubsection.}{0.5em}{}
\titleformat{\subsubsection}
  {\normalfont\normalsize\bfseries}
  {CHAPTER \thesubsubsection.}{0.5em}{}
\setlist{noitemsep}

% Header and footer
\pagestyle{fancy}
\fancyhf{}
\renewcommand{\headrulewidth}{0.4pt}
\renewcommand{\footrulewidth}{0.4pt}
\fancyhead[L]{\slshape Constitution}
\fancyhead[R]{
  \transparent{1}{\slshape Open Source Club}
  \transparent{0.7}{\includegraphics[height=10pt,width=10pt]{logo.png}}
}
\fancyfoot[C]{\thepage}

% Right-align Roman numerals in TOC
\usepackage{tocloft}
\renewcommand{\cftsecpresnum}{\hfill}
\renewcommand{\cftsecaftersnum}{\quad}
\setlength{\cftsecnumwidth}{3.5em}  % Still need adequate width for the largest numeral

% For subsections
\renewcommand{\cftsubsecpresnum}{\hfill}
\renewcommand{\cftsubsecaftersnum}{\quad}
\setlength{\cftsubsecnumwidth}{3em}

% For subsubsections (if you have them)
\renewcommand{\cftsubsubsecpresnum}{\hfill}
\renewcommand{\cftsubsubsecaftersnum}{\quad}
\setlength{\cftsubsubsecnumwidth}{3em}

% Document content starts here
\begin{document}

% Change bullet style
\renewcommand{\labelitemi}{\textcolor{accent}{\ding{68}}}
\renewcommand{\labelitemii}{\textcolor{accent}{\ding{67}}}
\renewcommand{\labelitemiii}{\textcolor{accent}{\ding{66}}}

% Title page
\begin{titlepage}
\begin{center}
\vspace*{1cm}

% Logo
\includegraphics[width=0.4\textwidth]{logo.png}\\[2cm]

% Title
{\LARGE\bfseries\color{accent} CONSTITUTION\\[0.5cm]}
{\huge\bfseries\color{accent} \underline{Open Source Club}\\[1cm]}

% Subtitle
{\large Last Revised: [Month Day, Year]}\\[3cm]

% Bottom of title page
\vfill
{\large Ohio State University}\\
{\large Columbus, OH}\\[1cm]
\end{center}
\end{titlepage}

% Table of contents
\tableofcontents
\clearpage

% Document proper starts here
\section*{PREAMBLE}
\addcontentsline{toc}{section}{PREAMBLE}

\begin{hangparas}{0.5in}{1}
  The Open Source Club at the Ohio State University has had an extensive role in OSU's
  tech-related student life, from sprouting into other clubs and events, to encouraging
  students and alumni alike to become a part of the open source community.

  The current leadership of this organization would like to extend gratitude to former
  members and staff; open source has had a rich history, both in and out of OSU's campus,
  and it is with the support of those who have come before us that we are able to exercise
  this passion so freely.

  A detailed record of past leadership for the Open Source Club at the Ohio State University
  is included in the complete legislature for the organization, which can be found at
  \href{https://legislature.osuosc.org}{legislature.osuosc.org}.

  Additionally, the following are the available direct links to the different ways in which
  this constitution can be read:

  \begin{itemize}
    \item \href{https://legislature.osuosc.org/constitution/preamble.html}{mdBook} (through
      your browser)
    \item \href{https://activities.osu.edu/involvement/student_organizations/find_a_student_org?i=1042}{PDF} (in the link provided to the right of "\textbf{Constitution}")
  \end{itemize}

  \textbf{This constitution is licensed as \href{https://creativecommons.org/licenses/by-sa/4.0/deed.en}{CreativeCommmons Attribution ShareAlike}. You are free to use it to create your own
  constitutions, both in format and in file, so long as you provide credit to the Open Source
Club at the Ohio State University for creating the original template and content, and allow
other individuals to do the same with whatever derivative constitution you create.}
\end{hangparas}

\section{Definition}

Article I of the Constitution for the Open Source Club at the Ohio State University specifies
the definition of the club, or, how the club ought to be referred to as, and the nature of the
club and its establishment.

\subsection{Name}

The name of this student organization is to be ``The Open Source Club at the Ohio State
University.'' ("The Open Source Club", "Open Source  Club", "OSC", "th(e/is) organization",
"th(e/is) student organization").

The following are appropriate shorthands that may be used throughout organization legislature
or in general reference to the organization:

\begin{itemize}
  \item The Open Source Club at OSU
  \item Open Source Club at OSU
  \item OSC at OSU
  \item OSC@tOSU
\end{itemize}

If the name is being provided in documentation or communications strictly within the
Ohio State University, suffixed terms "OSU" and "the Ohio State University" may be omitted
for brevity.

\subsection{Purpose}

This student organization will dedicate itself to helping promote, propagate, and standardize
the open source paradigm of technology development and use, both in software, and in other
works to which the open source paradigm is applicable and societally beneficial.

The organization will strive to excel in these three vectors by:

\begin{enumerate}[label=\textcolor{accent}{\alph*.}, leftmargin=2em]
  \item Introducing the open source model of development to OSU students and the larger
    OSU/Columbus community;
  \item Showcasing projects developed under an open source development model (programs,
    games, websites, arts, etc.);
  \item Fostering interest in the open source community, and providing ways for members
    to involve themselves in it (meetings, talks, workshops, etc.);
  \item Creating a space in which members can engage in constructive discourse on the
    topics surrounding open source;
  \item Raising awareness of the value the open source paradigm can provide to students,
    members of our communities, and society at large;
  \item Inspire members to think critically about the technology they use, the way that
    technology interfaces and affects their lives, and the ways in which we can reclaim
    our agency and foster solidarity in our communities through a more sustainable
    relationship with technology and a more democratic paradigm of technology in our societies.
\end{enumerate}

\subsection{Non-Discrimination Policy}

This organization does not discriminate on the basis of age, ancestry, color, disability,
gender identity or expression, genetic information, HIV/AIDS status, military status,
national origin, race, religion, sex, sexual orientation, protected veteran status, or
any other bases under the law, in its activities, programs, admission, and employment.

In addition, this organization does not discriminate on the basis of academic standing,
program affiliation, nor personal political affiliation.

Finally, it is this organization's policy that these statements and policies extend to
- and will be required of - all individuals who choose to associate with this
organization; student, faculty, or otherwise.

\section{Membership}

In the pursuit of our stated purpose, membership in this student organization is extended
to all members of the OSU and Columbus community who wish to involve themselves - and
learn how to benefit from - open source. No fees or dues will ever be collected from any
member of the community wishing to involve themselves in the Open Source Club.

Recognizing the breadth of our potential member pool, we define the two types of
members as \textbf{active} and \textbf{passive}, and elaborate on the privileges afforded
to each.

\subsection{Active Membership}

Active membership is awarded to any and all currently-enrolled students of the Ohio State
University, part- or full-time, who are in good standing with the organization, and who
display activity within the organization to a recognizable degree. This includes, but is
not limited to:

\begin{itemize}
  \item Those who attend two or more in-person meetings in a semester
  \item Those who are active in existing social groups (Discord, Revolt, Matrix, etc.)
    \begin{itemize}
      \item "Active" is loosely defined as contributing enough to online discussion so as
        to easily be precisely identified and recalled by members during in-person
        meetings, be it by username or reference to discussions of which the member in
        question was a part. A more precise definition is left to current officership
        discretion.
    \end{itemize}
  \item Other examples left to the officers' discretion.
\end{itemize}

Active membership affords members the privilege of continuing to participate in digital
social groups; being allowed to attend meetings and, to the discretion of associated
organizations, meeting collaborations.

Active membership additionally affords members the privilege of being able to run for,
and hold, executive officer positions; being able to vote for candidates for executive
roles; being able to participate in votes for the removal of an executive officer.

\subsection{Passive Membership}

Passive membership is awarded to any and all individuals who are in good standing
with the organization, and who display interest in the organization to a recognizable
degree. This includes, but is not limited to:

\begin{itemize}
  \item Those who subscribe to any existing mailing lists.
  \item Those who interact with any existing social media accounts.
  \item Those who are members of any existing social groups (Discord, Revolt, Matrix, etc.)
  \item Those who have expressed explicit interest in the club, but are unable to
    consistently interface with it due to personal constraints.
  \item Other examples left to the officers' discretion.
\end{itemize}

Passive membership affords members the privilege of continuing to participate in digital
social groups; being allowed to attend meetings and, to the discretion of associated
organizations, meeting collaborations.

\clearpage

\section{Procedures for Removing Members and Executive Officers}

It may be the case that a member of the organization, be they passive, active, or an
executive officer, has their behavior deemed incongruent with the stated goals of the
organization, in violation of the Ohio State University Standards, the Student Code of
Conduct, and/or the Non-Discrimination Policy as expressed in Article I, Section III, or
otherwise deemed unacceptable by the expectations of this organizations and its
membership. For these regrettable cases, and in the interest of fostering a productive,
accessible, and equitable community for the OSU community at large, a procedure for
removing the offending organization member is defined.

In calculating a non-unanimities (i.e. majorities or pluralities), abstentions from
active members are ignored. In calculating a unanimity, abstentions are equivalent to
nays.

Members removed from the organization through the procedures forthcoming lose their
status of good standing for two academic years.

\subsection{The Removal of a Non-Officer}

Non-officer members being considered for removal will be notified of this consideration,
whereafter a one-week voting period open to active members, passive members, and officers
will commence. The removal will be successful if:

\begin{itemize}
  \item A unanimous decision to remove is made among the officers.
  \item A simple majority decision to remove is made among passive and active members, separately.
\end{itemize}

\subsection{The Removal of an Officer}

Officer members being considered for removal will be notified of this consideration,
whereafter a two-week voting period open to active members will commence. The removal will
be successful if:

\begin{itemize}
  \item A \sfrac{2}{3} majority decision to remove is made among active members.
\end{itemize}

\section{Executive Officership}

In the pursuit of effective organization management, a structure of executive officership
is designed for active members of the organization to play a role in maintaining the
community and longevity thereof.

\clearpage

\subsection{Titles and Responsibilities}

Any role not marked with an asterisk (*) is considered supplementary, and its being
unfulfilled during an academic term is acceptable. In such an event, the available
executive officership will take on the stated responsibilities at their discretion. As a
default, any role not fulfilled will have their associated responsibilities taken on by
the President and Vice President as appropriate.

All Executive roles will have alternative titles that are to be recognized by past
leadership and advisors as valid. They exist so as to allow flexibility in how members of
the Executive Board of OSC represent their Officership to others, be that in professional
documents, in communicating with non-OSC parties, or otherwise.

\subsubsection{*President}
The President is the primary leader of the executive board, and oversees the large-scale
success of the organization.
\begin{itemize}
  \item The following is a non-exhaustive list of the expected responsibilities of
    the President of Open Source Club:
    \begin{itemize}
      \item Deciding meeting times and topic;
      \item Maintaining appropriate and welcoming behavior in OSC spaces (in-person,
        digital, and otherwise);
      \item Establishing and/or disestablishing social media accounts and virtual
        social groups;
      \item Representing OSC in matters with non-OSC parties;
      \item Ensuring detailed records and recommendations of OSC procedure are
        maintained;
      \item Ensuring all training and documents are completed and filled by all
        members of the Executive Board by time frames places by the Office of Student
        Life of the Ohio State University;
      \item Providing a model for what an effective, inclusive, and mindful member of
        OSC ought to behave like;
      \item Being mindful of the assets and events presently and in the future under
        OSC's purview.
    \end{itemize}
  \item Acceptable alternative titles:
    \begin{itemize}
      \item Benevolent Dictator
      \item Organization Lead
      \item Primary Czar
    \end{itemize}
\end{itemize}

\subsubsection{Vice President}
The Vice President is the secondary leader of the Executive
Board, and ought to support the President in their responsibilities as both parties see
fit and appropriate.
\begin{itemize}
  \item The following is a non-exhaustive list of the expected responsibilities of the
    Vice President of Open Source Club:
    \begin{itemize}
      \item Assisting the President in their responsibilities where the President feels
        they require assistance;
      \item Maintaining communications with non-OSC parties as deemed appropriate;
      \item Raising awareness and interest in OSC as deemed appropriate.
    \end{itemize}
  \item Acceptable alternative titles:
    \begin{itemize}
      \item Supporting Organization Lead
      \item Secondary Czar
      \item Vice President of Operations
    \end{itemize}
\end{itemize}

\subsubsection{*Treasurer}
The Treasurer is the Executive Officer tasked with overseeing
and managing existing financial assets belonging to OSC, as well as in ensuring
that OSC is maintaining sustainable pecuniary practices. They are expected to maintain
detailed records of all assets and funds belonging to OSC, as well as a detailed history
of pecuniary transactions.
\begin{itemize}
  \item The following is a non-exhaustive list of the expected responsibilities of the
    Treasurer of Open Source Club:
    \begin{itemize}
      \item Maintaining detailed records of transactions involving OSC in all matters
        that would affect the financial state of the organization;
      \item Maintaining up-to-date records on the digital, physical, and financial assets
        belonging to OSC;
      \item Preparing end-of-semester financial statements for OSC;
      \item Ensuring all members of OSC who, in immediate lack of access to OSC funds
        used ``out-of-pocket'' funds for direct OSC matters, are appropriately reimbursed,
        and that such reimbursements are effectively recorded in financial documentation.
    \end{itemize}
  \item Acceptable alternative titles
    \begin{itemize}
      \item Financial Lead
      \item Financial Czar
      \item Vice President of Finance
    \end{itemize}
\end{itemize}

\subsubsection{Director of Software Project Development}
The Director of Software Project
Development is the Executive Officer tasked with fostering an inter-supportive community
of software and/or hardware developers where members can improve their skills in
making technical contributions to open source and developing their own open source
projects, as well as find a community of like-minded developers to help one another
become more involved in the technical side of the open source community at large. 
\begin{itemize}
  \item The following is a non-exhaustive list of the expected responsibilities of the
    Director of Software Project Development of Open Source Club:
    \begin{itemize}
      \item Independently leading meetings and, optionally, a dedicated development
        subteam of OSC focused on software development within the larger open source
        community;
      \item Act as a central hub for information regarding involving oneself with
        open source development for members.
    \end{itemize}
  \item Acceptable alternative titles:
    \begin{itemize}
      \item Technical Development Czar
      \item Software Development Czar
      \item Development Czar
      \item Vice President of (Software) Development
      \item Secretary of (Software) Development
    \end{itemize}
\end{itemize}

It is important to recognize that any given incarnation of OSC leadership may feel that
some of these responsibilities are better handled by members of the Executive board
different from the one to whom they are described to belong in this Section. As a rule of
thumb, shifting of responsibilities is acceptable so long as the stated general-purpose
role of the officer(s) in question is neither infringed upon nor ignored, and so long as
all officers in their roles feel they have a set of responsibilities representative of
the workload suggested by the responsibilities provided herein.

\subsection{Voting and Tenure}

Elections are to be held annually as per the procedures provided herein.

All voting procedures occur within a time frame defined by the first weekend after the
last day of proper classes (i.e. not finals). For brevity, we refer to this weekend as
the Post-class Weekend. All events and time periods that are defined to start some
"weeks" before this Weekend are implied to have thresholds at Sunday, 11:59 PM.

\subsubsection{Time Allotted to Self-Nomination and Voting}

7 weeks exactly before the Post-class Weekend, a three-week period of self-nomination will
open. No less than one week before this period, current Executive leadership will have
notified active members at large of the upcoming self-nomination period. Current Executive
leadership is to prepare a manner in which members can self-nominate for all Executive
roles and potential new ones, should they feel that there is a set of responsibilities
warranting their own dedicated role, and that they are qualified to fulfill them. The
creation of this new role is left to current Officer discretion as established in Section
III of this article. Whatever manner of self-nomination is selected must also include a
way for self-nominees to justify their candidacy for the role (be it past experience,
relevant skills, etc.) so that voters are better prepared to gauge the available candidates.
An active member is permitted to self-nominate for more than one Executive position, but
is only permitted to hold one Executive position during their tenure. Candidates are to
disclose Executive position preference as a part of their self-nomination. Self-nomination
is to close 4 weeks before the Post-class Weekend.

As soon as active members are notified of the candidates, candidates are free to campaign
for their positions, so long as their campaigning is congruent with the expected behaviors
of an OSC member in good standing. 3 weeks before the Post-class Weekend, a three-week
period of voting will open that ends on the threshold of the Post-class Weekend.

No later than 2 weeks after the Post-class Weekend, current Executive leadership is to
announce the results of the election.

\subsubsection{Precise Voting and Vote-Counting Procedure}

When the voting time period has opened, voting is to be completed using rank-choice for
each Executive role. All candidates for a given role are to be listed, and voters are to
list them in order of preference.

In counting these votes, the first candidate to reach 50\% first-rank approval is awarded
the position. If no candidate receives 50\% first-rank approval, the candidate with the
least first-rank approvals is eliminated from the count, and the votes allotted to them
are divided among the next-highest-ranked candidate for each voter. First-rank approval
is recalculated in this manner until a candidate receives 50\% of the votes.

If a single candidate is awarded multiple Executive positions, their highest-preferred role
is awarded to them. For the role(s) they were not awarded, the votes are recounted with
all candidates back in the pool, but with the candidate in question removed. The winner
of the role is decided by the preceding process once again.

This process is to repeat itself until all candidate-available roles are filled by unique
candidates.

\subsubsection{Tenure}

The winning candidates are appointed to their roles on the first day of the August
immediately succeeding the election period. Current Executive leadership is expected to
facilitate the adoption of Executive responsibilities justly and amicably during the
period in which the new Executive leaders have been announced but not yet appointed.
These candidates hold their roles until the candidates awarded Executive positions in the
next annual election cycle are appointed. That is, Executive officership is granted for
one year, from August 1st to August 1st.

Members of OSC are permitted to run for Executive Officership so long as they meet the
requirements of an active member as provided in Article II, Section I.

\subsection{Modifying Officer Positions}

As expressed in Section I of this Article, this organization must always have unique
individuals fulfilling the following roles:

\begin{itemize}
  \item President
  \item Treasurer
\end{itemize}

In the case of substantial changes to organization membership, current Officers may deem
it reasonable to create new Officer positions, or eliminate existing ones. When this
occurs is left entirely to Officer discretion, with the details surrounding procedure of
accepting changes to Executive Officership being left to current Executive Officers upon
which to decide.

In the event of an added role, the role in question is to be logged in the Office of
Student Life portal as a "Secretary." The precise name is left to current Executive
Officership to decide upon, and is to be entered in the field designated "Other Title."
The only exception to this rule is for the additional role of Vice President, which
should be labeled in the system as such. Historically, the naming convention has been as
follows, where one is selected as the "primary" title, and others are usable by members
who hold the role, as expressed in Article IV, Section I:

\begin{itemize}
  \item Director of [RESPONSIBILITY]
  \item Vice President of [RESPONSIBILITY]
  \item {[RESPONSIBILITY]} Lead
  \item {[RESPONSIBILITY]} Czar
  \item Secretary of [RESPONSIBILITY]
\end{itemize}

These are suggestions, and they can be eschewed in favor of a title agreed upon by current Executive Officership, including the active member who would be adopting the role. Any changes made to Executive Officership must be made public to membership prior to the role changes taking effect.

\section{Advisor Selection}

Advisors of student organizations must be full-time members of the University faculty or
Administrative \& Professional staff. Advisors are expected to approve any changes made to
the Constitution, Goals, or other legislature of the organization. They are additionally
expected to approve programming and meeting decisions, as well as budgeting decisions.
Advisors are expected to meet with the officers of the organization at least once per term.

Finally, advisors are encouraged to meet with the officers on a more regular basis, such
as once per month, as well as give general opinions and advice pertaining to the success
of the organization.

\section{Organization Meetings}

While meeting attendance is not explicitly required to be a member of this organization
(excepting active membership), it is highly encouraged. In-person meetings shall occur
no less than two times in a semester, pending extenuating public health considerations,
individual, communal, municipal, or otherwise.

It is a stated goal of the OSC to perform collaborative meetings with other
organizations. Whether and which of these meetings are considered official OSC meetings
is left to the discretion of the current officers.

\section{Management of Organization Assets}

Whether they be in active us or not, all assets currently belonging to the Open Source
Club must be kept recorded as such by the Treasurer, as stated in their responsibilities
in Article IV, Section I. It is left to current Executive Officership how and whether
these assets are utilized, but it is paramount that their existence be made clear for
future Executive Officers.

\subsection{Physical Assets}

All physical assets are to be stored and cared for in a location agreed upon by the current
officers.

\subsection{Digital Assets}

In addition to other assets that are typically held by OSU organizations, the Open Source
Club will also manage a GitHub organization.

At any given time, the only individuals who are permitted write access to the GitHub
organization and all repositories over which it presides are the three primary officers
labeled in Article IV Section III, the faculty advisor(s) of the organization, and any
other additional officers whose explicitly-named duties in this constitution include the
management of this organization's digital assets. It is the responsibility of those
aforementioned to ensure that, when the individuals taking on these roles are changed,
permissions are revoked and granted properly within one month of the decision for these
roles to be changed.

Non-write access is left to the current officers' discretion.

\section{Methods of Amending This Constitution}

Proposed amendments shall be presented to the other officers and then discussed among
them. If a majority of officers approve of the amendment, it can be made to the constitution.

\section{Methods of Dissolving this Organization}

For this organization to dissolve, a unanimous vote among the officers to do so shall
take place. Should any assets and/or debt exist, the advisers shall see fit to dispose of
it.

% Signature section
% \clearpage
% \section*{RATIFICATION}
% \addcontentsline{toc}{section}{RATIFICATION}
%
% We, the undersigned, hereby ratify this Constitution on behalf of the Open Source Club at
% the Ohio State University.
%
% \vspace{1cm}
% \begin{tabular}{p{3in}p{3in}}
% \rule{3in}{0.5pt} & \rule{3in}{0.5pt} \\
% President & Faculty Advisor \\
% & \\
% \rule{3in}{0.5pt} & \rule{3in}{0.5pt} \\
% Vice President & Date of Ratification \\
% \end{tabular}
%
\end{document}
